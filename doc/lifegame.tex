\documentclass{jsarticle}
\begin{document}

\title{ライフゲームを作ろう}
\author{宇佐美雅紀}
\maketitle

Haskellの練習として、ライフゲームを作成します。

\section{ライフゲームとは}
ライフゲームは、生命の誕生、進化、淘汰などのプロセスを簡易的なモデルで再現したシミュレーションゲームであり、セル・オートマトンの一種です。
詳しくは、Wikipediaでライフゲームを検索してみてください。( http://ja.wikipedia.org/wiki/%E3%83%A9%E3%82%A4%E3%83%95%E3%82%B2%E3%83%BC%E3%83%A0 )

\section{ライフゲームのルール}
\subsection{誕生}
死んでいるセルに隣接する生きたセルがちょうど3つあれば、次の世代が誕生する。
\subsection{生死}
生きているセルに隣接する生きたセルが2つか3つ以下ならば、次の世代でも生存する。
\subsection{過疎}
生きているセルに隣接する生きたセルが1つ以下ならば、過疎により死滅する。
\subsection{過密}
生きているセルに隣接する生きたセルが4つ以上ならば、過密により死滅する。

\end{document}
